\documentclass[12pt]{article}
\usepackage[utf8]{inputenc}
\usepackage[a4paper, total={6.5in, 8in}]{geometry}
\usepackage{amsmath}
\usepackage{amssymb}
\usepackage{float}
\usepackage{xcolor}
\usepackage{graphicx}
\usepackage{hyperref}
\parindent 0 px

\title{\textbf{\Huge{\hrule\hrule\textcolor{black}{${}$\\ Basic Group Theory}\\[12pt]\hrule\hrule}}}\author{Nripendra Kumar Deb }
\date{November 2022}

\begin{document}

\maketitle

\textbf{\textcolor{violet}{Definition:}}\\
Normaliser of a subgroup $H$ of a group $G$ is defined as $$N(H)=\{g\in G\left.\right| gHg^{-1}=H\}$$
\textbf{\textcolor{violet}{Definition 2 :}}\\
Consider $S$ to be the set of all conjugates of $H$. Consider the \emph{stabilizer} of $H$ under the conjugation action of $G$ :$$\text{stab}(H)=\{g\in G\left.\right| gHg^{-1}=H\}$$
So $N(H)=\text{stab}(H)$ under the conjugation action of $G$ on $S$.\\
$\bullet$ Further note that $H\subset N(H)$ and $H$ is normal in $N(H)$, hence the name \emph{normalizer}.\\
\textbf{Remark}: Both definitions have useful applications.\\

\textbf{\textcolor{violet}{Lemma 1 :}}\\
If G be a group such that $|G|=p^km$, and let $S$ be the set of all $p-sylow$ subgroups with $P\in S$ be fixed then $$|S|=[G:N(P)]$$
\textbf{\emph{Proof :}}
Firstly, recall that any two $p-sylow$ subgroups of a group $G$ are conjugates of each other and any conjugate of a $p-sylow$ is agin a $p-sylow$(as conjugates have same cardinality). Thus $S$ is in fact the set of all conjugates of $P$.
Consider the conjugation action of $G$ on $S$. Consider any two $p-sylow$ subgroups $P_1$ and $P_2$ of $G$. Since $P_1$ and $P_2$ are conjugates of each other so $\exists g \in G$ s.t $gP_1g^{-1}=P_2$ or $g.P_1=P_2$\footnote{Here $'g.'$ represents the action} , which proves the \emph{transitivity} of the action.\\
Let $P\in\mathcal{O}$ where $\mathcal{O}$ is an orbit, since the action is \emph{transitive} $\mathcal{O}=S$ . From the orbit stabilizer theorem we have $|\mathcal{O}|=\frac{|G|}{|G_p|}$ i.e $$|\mathcal{O}|=[G:G_P]$$
Here $G_P$ is the \emph{stabilizer} of P. Now using definition 2 we get $G_P=N(P)$ which gives
$$|\mathcal{O}|=|S|=[G:N(P)]$$
\textbf{\textcolor{violet}{Lemma 2 :}}\\
If $H_1\subset H_2\subset G$, then $[G:H_2]$ divides $[G:H_1]$.\\
\textbf{\emph{Proof :}} Let $[G:H_1]=m$ and $[G:H_2]=n$. Using counting formulae we have
$$|G|=m|H_1|$$$$|G|=n|H_2|$$
using these we get
$$\frac{|H_2|}{|H_1|}=\frac{m}{n}$$
Lagrange Theorem tells that $|H_1|$ divides $|H_2|$ and hence $n$ divides $m$.

\textbf{\textcolor{violet}{Theorem 1 :}}\\
If G be a group such that $|G|=p^km$, and let $S$ be the set of all $p-sylow$ groups then
$|S|$ divides $m$.\\[8pt]
\textbf{\emph{Proof :}} Fix one $P\in S$. Using \emph{Lemma 1} we have $$|S|=[G:N(P)]$$
Note that $P\subset N(P)$ and $[G:P]=m$ hence invoking \emph{lemma 2} we get $|S|$ divides $m$.\\
\textbf{\textcolor{violet}{Lemma 3 :}}
Let $G$ be a $p-group$ such that $G$ acts on $X$ and $X^{G}$ is the set of all fixed points of $X$, then
$$|X^{G}|\equiv |X|\;(\text{mod}\;p)$$
\textbf{\textcolor{violet}{Corollary:}} If $p$ does not divide $|X|$, then $X$ has a fixed point.\\
\textbf{\emph{Proof :}} Using \emph{lemma 3} we have $$|X^{G}|\equiv |X|\;(\text{mod}\;p)$$
i.e $p$ divides $|X^{G}|-|X|$ and as $p$ does not divide $|X|$ so $p$ must not divide $|X^{G}|$(why?).\\
Since $P$ does not divide $|X^{G}|$ hence $|X^{G}|>0$ or $X^G$ is non-empty, we are done.\\
\textbf{\textcolor{violet}{Theorem 2 :}}\\
If G be a group such that $|G|=p^km$, and let $S$ be the set of all $p-sylow$ groups then $$|S|\equiv 1\; (\text{mod}\; p)$$
\textbf{\emph{Proof :}} 
Note that setting $X=S$ as in \emph{lemma 3} leaves us to show that $S$ has a unique fixed point or $|X^G|=1$. One can easily notice that $P\in S$ is a $p-group$. So we take action of $P$ on the set $S$ via conjugation. Let $Q$ be a fixed point in $S$ then $gQg^{-1}=Q$ for all $g\in P$ which implies $P\subset N(Q)$. Now note that $Q$ is normal in $N(Q)$ and so $Q$ is a unique $p-sylow$ in $N(Q)$ but $P$ is also a $p-sylow$ in $N(Q)$\footnote{Since $P$ and $Q$ both belong to $S$ so $|P|=|Q|$} thus $P=Q$, which proves that the fixed point is unique. Hence we get $$|S|\equiv 1\; (\text{mod}\; p)$$

\textbf{\textcolor{red}{Exercise}} Let $N$ be a nontrivial normal subgroup
of a $p$-group $G$. Show that $N$ must intersect the center
of $G$ non trivially.\\
\textbf{\textcolor{violet}{Solution :}}\\
Let's take the conjugation action of $G$ on $N$ . The action is justified because the subgroup $N$ is normal. Using Class equation we have:
$$|N|=|Z_{N}(G)|+\sum_{|{\mathcal{O}_i|>1}}\left|{\mathcal{O}}_{i}\right|$$
Here $Z_{N}(G)$ consists of the single orbits i.e let $\{x\}\in Z_{N}(G)$ then 
$x\in N$ and $gxg^{-1}=x\;\forall g\in G$ . In fact note that if $Z(G)$ is the center of the group then $$N\cap Z(G)=Z_{N}(G)$$
Note that $p$ divides $|Z_{N}(G)|$ so $$|N\cap Z(G)|=|Z_{N}(G)|>0.$$
\textbf{\textcolor{red}{Exercise}}
Let $G$ be a $p$-group and let $p^k$ be a divisor of
$|G|$. Show that $G$ has a subgroup of order $p^k$.\\
\textbf{\textcolor{violet}{Solution :}}\\
Let's assume $|G|=n$ we will do induction on $n$. For $n=1$, $G$ is a cyclic group, so there exists elements of order $1$ and $p$. Hence base case holds.
Suppose that it holds for $|G|\in\{1,....,p^{n-1}\}$. Consider a group $G$ such that $|G|=p^n$.
Since $G$ is a $p$-group so $C$ is non-trivial. Let $|C|=p^{l}$ now consider the group $G/C$(Why is this a group?). Then $|G/C|=p^{n-l}$ , by induction hypothesis there exist a subgroup $H/C$ $(H\subset G)$ such that $|H/C|=p^{k-l}$ as $k\leq n\Rightarrow k-l\leq n-l$. So
$$|H|=|C|\times p^{k-l}\Rightarrow |H|=p^{k}.$$
\textbf{\textcolor{red}{Exercise}}
There are 6 subsets of order 2 of $\{1, 2, 3, 4\}$.
Any element of $S_4$ permutes these subsets.Show that
the resulting homomorphism from $S_4$ to $S_6$ is injective
and its image lies in $A_6$.\\
\textbf{\textcolor{violet}{Solution :}}\\
Let $\sigma\in S_4$ and we define $f:S_4\mapsto S_6, f(\sigma)((i,j))=(\sigma(i),\sigma(j))$ . $$f(\sigma_1\sigma_2)((i,j))=(\sigma_1\sigma_2(i),\sigma_1\sigma_2(j))= f(\sigma_1)((\sigma_2(i),\sigma_2(j)))=f(\sigma_1)f(\sigma_2)((i,j))$$$$\Rightarrow f(\sigma_1\sigma_2)=f(\sigma_1)f(\sigma_2)$$
Hence $f$ is a homomorphism. Let $f(\sigma_1)=f(\sigma_2)$ then $\sigma_1=\sigma$ follows directly (Check!!). So $f$ is injective.
Note that if $\sigma_1$ and $\sigma_2$ both belong to same conjugacy class then 
$\sigma_1=\sigma\sigma_2\sigma^{-1}$ then $f(\sigma_1)=f(\sigma_2)$. So we have to check the image for one element per conjugacy class. Note that a conjugacy class consists of elements one same cycle type exactly. Check that the images lies in $A_3$.\\
\textbf{\textcolor{red}{Exercise}}
Show that there are 36 Sylow 5-subgroups
in $A_6$.\\
\textbf{\textcolor{violet}{Solution :}}\\
Note that $|A_6|=120=2^2.3^2.5$. And order of $5-$sylow subgroup is 5 so it is a cyclic group note that each 5-sylow contains 4 elements of order 4 so if total number of 5-sylow subgroups is $m$ . Then total number order 5 elements is $A_6$ is $4m$ as other only 5-sylow subgroups contributes to order 5 elements.
Order 5 element in $A_6$ is of cycle type $5^11^1$ which is same as numbers of elements of cycle type $5^11^1$, hence
$$4m=\frac{6!}{5^11^1.1!.1!}\Rightarrow 4m=144\Rightarrow m=36.$$
\textbf{\textcolor{red}{Exercise}}
Partition $\{1, . . . , 6\}$ into two subsets S
and $T$ of order 3. Let P be the set of elements in $A_6$
which permute S and T either trivially or by a 3-cycle.\\
a) Show that $|P| = 9$ and $P$ is a 3-Sylow subgroup.\\
b) Prove that each 3-Sylow subgroup of $A_6$ is of the
above form.\\
c) Deduce that there are 10 3-Sylow subgroups of $A_6$.\\
\textbf{\textcolor{violet}{Solution :}}\\
(a) Let $\{a,b,c\}$ and $\{d,e,f\}$ be two such subsets. Note that $(abc)$,$(acb)$ are the only possible 3-cycles of $P$ which permute $\{a,b,c\}$ so there are $3$ elements in P which permutes $\{a,b,c\}$ and similarly for $\{d,e,f\}$ . So in total we have $3\times 3=9$ elements. ( Here it is assumed that the action of $A_6$ is taken on the whole set which in fact gives a permutation of the subsets). Since $A_6=2^2.3^2.5$ hence $P$ is a 3-sylow subgroup.\\[4 pt]
(b) Since two p-sylows are conjugates of each other. So any 3-sylow subgroup will be a conjugate of the above subgroup and as  conjugates have same cycle type so every element of any other 3-sylow will have same cycle type as that of the elements of the above subgroup and hence of that form.\\[4 pt]
(c) Since every 3-sylow subgroup is of the above form so the number of 3-sylow subgroups 
are basically the number of ways to select the two subsets. Note that the order of selecting the subsets doesn't matter. So total number of 3-sylow subgroups is $\frac{\left(^{6}_{3}\right)}{2!}=10.$\\
\textbf{\textcolor{red}{Exercise}}
Show that there are $(p-2)!$ Sylow p-subgroups of $S_p$(p is a prime). Deduce that:$$(p-1)!\equiv -1\;(\text{mod}\;p)$$
\textbf{\textcolor{violet}{Solution :}}\\
We have $|S_p|=p!=p(p-1)!$ so p-sylow subgroups of $S_p$ has order $p$ as its maximum power of $p$ dividing $|S_p|$. So every p-sylow subgroup has $(p-1)$ elements of order $p$ as they are cyclic. So total $n_p(p-1)$ order $p$ elements are there.\\
Again elements of order $p$ in $S_p$ are precisely the $p$-cycles and number of $p$-cycles in $S_p$ is $\frac{p!}{p}=(p-1)!$. Hence we get $$n_p(p-1)=(p-1)!\Rightarrow n_p=(p-2)!$$
Now using sylow's theorem and a bit of modular arithmetic we get 
\begin{align*}
&\;\; \;\;\;\;\;\;n_p  \equiv 1\;(\text{mod}\;p)\\ & \Rightarrow (p-2)!\equiv 1\;(\text{mod}\;p) \\ 
& \Rightarrow (p-1)(p-2)!\equiv (p-1)\;(\text{mod}\;p) \\ & \Rightarrow (p-1)!\equiv -1\;(\text{mod}\;p) 
\end{align*}
\textbf{\textcolor{red}{Exercise}}
Let $H$ be a subgroup of a $G$. If $H$ is a p-group, show that $H$ is contained in a p-Sylow
subgroup of $G$.\\
\textbf{\textcolor{violet}{Solution :}}\\
Consider a p-sylow subgroup $P$ and the set $G/P$. Act $H$ on this set via left multiplication.
Note that if $X$ is a set and a p-group acts on $X$ then if $p$ does not divide $|X|$ then $X$ has a fixed point\footnote{Use $|X|\equiv |X^G|\;(\text{mod}\;p)$}. So here $G/P$ has a fixed point $gP$ say. Then $\forall\;h\in H$ we have $$hgP=gP\Rightarrow g^{-1}hgP=P\Rightarrow g^{-1}hg\in P\Rightarrow h\in gPg^{-1}$$
Thus we get $H\subset gPg^{-1}=P'$, note that $P'$ is also a p-sylow subgroup as cardinality of conjugate subgroups are equal.\\
\textbf{\textcolor{red}{Exercise}}
Let $H$ be a subgroup of $G$ and let $P$ be a p-Sylow subgroup of $G$.\\
a) Show that $H \cap P'$ is a p-Sylow subgroup of $H$ where
$P'$ is a conjugate of $P$.\\
b) Show that $G$ can be embedded in $GL_n(\mathbb{F}_p)$ for
suitable n.\\
\textbf{\textcolor{violet}{Solution :}}\\ (a) Consider the similar set and action as the previous question. Since $p$ does not divide $|G/P|$ so by class equation we know there exists some orbit $\mathcal{O}$ of $H$ such that $p$ does not divide $|\mathcal{O}|$ and so $|\mathcal{O}|=1$.
Let $gP\in \mathcal{O}$ then by Orbit stabilizer theorem we get $$|H|=|stab(gP)||\mathcal{O}|$$
Note that 
\begin{align*}
stab(gP) &=\{h\in H\left.\right|hgP=gP\}\\ &=\{h\in H\left.\right|g^{-1}hgP=H\}\\&=\{h\in H\left.\right|g^{-1}hg\in P\}\\&
=\{h\in H\left.\right|h\in gPg^{-1}\}=\{h\left.\right|h\in (H\cap gPg^{-1})\}\\\\
\end{align*}
Hence we get $stab(gP)=H\cap P'$ , here $P'=gPg^{-1}$ and since  $H\cap P'\subset P'$ so $|H\cap P'|$ divides $|P'|$ and thus $|H\cap P'|=p^m$ again $p$ does not divide $|\mathcal{O}|$ so $H\cap P'$ is a p-sylow in $H$.\\[4 pt]
(b) \textbf{Just an Attempt} \\If $|G|=n$ then using Cayley's Theorem we know that $G$ can be embedded in $S_n$.
Let $\beta=\{\epsilon_1,....,\epsilon_n\}$ be the standard basis of $\mathbb{R}^n$. And for $\sigma \in S_n$ let $T_{\sigma}(\epsilon_i)=\sigma(\epsilon_i)$ here $\sigma$ permutes $\beta$ , $P_{\sigma}$ be matrix representation of $T_{\sigma}$.
Now consider the map $$f:S_n\to GL_n(\mathbb{F}_p),\qquad \sigma\mapsto P_{\sigma}$$ 
Note that this map $f$ is an isomorphism, so $s_n$ can be embedded in $GL_n(\mathbb{F}_p)$. Hence indeed $G$ can be embedded in $GL_n(\mathbb{F}_p)$.\\[4 pt]
(c) Note that $GL_n(\mathbb{F}_p)$ has a $p-$sylow subgroup consisting of upper triangular  matrices with diagonal entries $1$. Let $P$ be one such p-sylow subgroup, and $Im(G)$ is the image of $G$ embedded in $GL_n(\mathbb{F}_p)$ then using part (a) $P'\cap Im(G)$ has a p-sylow subgroup in $Im(G)$,taking the pre-image of that subgroup gives a p-sylow subgroup of $G$.\\
\textbf{\textcolor{red}{Exercise}}
Let $N$ be a normal subgroup of $G$. If $P$
is a Sylow subgroup of $G$, show that $N\cap P$ is a Sylow
subgroup of $N$.\\
\textbf{\textcolor{violet}{Solution :}}\\ Using previous exercise we know that there exists some $P'$
such that $N\cap P'$ is a p-sylow subgroup in $N$. We claim that $N\cap P=N\cap P'$ . It follows
$$N\cap P'=N\cap gPg^{-1}=gNg^{-1}\cap gPg^{-1}\stackrel{?!}{=}g(N\cap P)g^{-1}$$
Note that conjugate of sylow group is a sylow group( check cardinality). Hence $N\cap P$ is a p-sylow subgroup of $N$.\\
\textbf{\textcolor{red}{Exercise}} let $f:G\to G'$ be a surjective homomorphism. If $P$ is a sylow subgroup of $G$ then $f(P)$ is also a sylow subgroup of $G'$.\\
\textbf{\textcolor{violet}{Solution :}}\\
Consider $G=p^k.m$ such that $gcd(p,m)=1$ and $|P|=p^k$ so $[G:P]=m$.
Since $f$ is a sujective homomorphism so $Im(f)=G'$. Using First isomorphism theorem\footnote{If $f:G\to G'$ is a homomorphism then $G/ker(f)\cong Im\;f$} we get $$|G|=|ker(f)||G'|$$
If we restrict the domain of $f$ to $P$ we get
$$|P|=|ker(f)\cap P||f(p)|$$
We get $$\frac{[G:P]}{[G':f(P)]}=\frac{|G||f(P)|}{|G'||P|}=\frac{|ker\;f|}{|ker(f)\cap P|}$$
By Lagrange's theorem $|ker(f)\cap P|$ divides $|ker\;f|$ so $[G':f(P)]$ divides $[G:P]$.
Since $|f(P)|$ divides $|P|$ so $f(P)$ is a p-group now we claim that $gcd([G':f(P)],p)=1$.\\ Indeed $[G':f(P)]$ is a factor of $[G:P]$ and $[G:P],p$ are co-prime to each other so $[G':f(P)]$ and $p$ are also co-prime, which proves the fact that $f(P)$ is a sylow subgroup of $G'$.\\[4 pt]
\textbf{\textcolor{red}{Exercise}}
Let $H, H'$ be two subgroups of $G$ such that $H'$ normalizes $H$ (i.e., $H'$ is contained in $N(H)$).
Show that $HH'$ is a subgroup of G.\\
\textbf{\textcolor{violet}{Solution :}}\\
We will use one step subgroup test here. Let $h_1h_1'$ and $h_2h_2'$ be two elements in $HH'$.
Now 
\begin{align*}
h_1{h_1}'(h_2{h_2}')^{-1}&=h_1{h_1}'({h_2}')^{-1}h_2^{-1}\\ &=h_1{h_1}'({h_2}')^{-1}h_2^{-1}{h_2}'({h_2}')^{-1}\\& =h_1{h_1}'h({h_2}')^{-1}\\& =h_1{h_1}'h({h_1}')^{-1}{h_1}'({h_2}')^{-1}\\
&=h_1h_{0}{h_1}'({h_2}')^{-1}\\&=h_1h_0{h_3}'
\end{align*}

here $h=({h_2}')^{-1}h_2^{-1}{h_2}'\in H$ as $H'$ normalizes $H$.
So $h_3'\in H'$ and  $h_1{h_1}'(h_2{h_2}')^{-1}=h_1h_0h_3'\in HH'$.\\[4 pt]
\textbf{\textcolor{red}{Exercise}} Let $N$ be the normalizer of a Sylow p-
subgroup $P$ of $S_p$. Show that $|N|=p(p-1).$\\
\textbf{\textcolor{violet}{Solution :}}\\
Let the $S$ be the set of all p-sylows of $S_p$ then using the definition 2 of normalizer and the proof of \textbf{lemma 1} we get
$$|S|=[S_p:N]$$
Now in one of the previous exercise we determined $|S|=(p-2)!$, hence it follows that
$$(p-2)!=\frac{|S_p|}{|N|}\Rightarrow |N|=\frac{p!}{(p-2)!}=p(p-1)$$
\end{document}
